\documentclass{article}

\usepackage{amsmath,amssymb,amsthm}
%%% Don't mess with this file!
\usepackage[utf8]{inputenc}
\usepackage{fullpage}
\usepackage{enumitem}

\setlist[enumerate]{itemsep=3pt,topsep=3pt}
\setlist[enumerate,1]{label=(\alph*)}

\renewcommand\baselinestretch{1.5}

\newcommand{\WAtitle}[2]{\noindent\textbf{\Large Writing Assignment \#{#1} \hfill due #2}\\}
\newcommand{\WAauthors}[1]{\noindent {#1}\\}
\newcommand{\defi}[1]{\textbf{\textit{#1}}}

\theoremstyle{definition}
\newtheorem*{defn*}{Definition}
\newtheorem*{prop*}{Proposition}
\newtheorem*{example*}{Example}
\newtheorem*{thm*}{Theorem}

\DeclareMathOperator{\Row}{row}
\DeclareMathOperator{\Col}{col}
\DeclareMathOperator{\trace}{Tr}

\begin{document}

\WAtitle{1}{Wednesday, August 26} %% Put the assignment number in the first { } and the due date in the second { }
\WAauthors{Prof.\ Gibbons and Cats}{Myrka Aguilar-Montano} %% Put the author(s) here

\begin{prop*}[Autobiography Proof] 
    For all real numbers $x$ and $y$, if $x+y = 9$ and $2x-y = 6$, then $x = 5$ and $y = 4$. 
\end{prop*}

\begin{proof}
    Let $x$ and $y$ be real numbers. Consider the linear system of equations
    \begin{align}
        x+y &= 9 \label{eq1} \\
        2x-y &= 6 \label{eq2}.
    \end{align}
    
    % Explain how you got to your solution in your autobiography. For instance, you might start with (remove the "%" to uncomment the next line):
    % Solving Equation~\eqref{eq1} for $y$ we obtain $y = 9-x$.
    % But if you did something different in your autobiography, you'll want to do that instead!
    % To reference an equation, use \eqref{}.
    Taking equation \eqref{eq1} and multiplying it by (-2) while solving for $y$ yields $ y=4$ through the process of elimination. Using $y=4$, we substitute it into either equations in search of $x$, leading to an answer of $x=5$. Therefore, given the linear system of equations \eqref{eq1} and \eqref{eq2}, $x=5$ and $y=4$ for all real numbers $x$ and $y$.
    
    
    
    
    
    
    
    
\end{proof}

\end{document}

There are useful pieces of code in the file WA-0.tex -- copy and paste them into the document above to get started. If you run into any trouble, or have LaTeX questions, share your file with Prof. Gibbons and ask for some assistance.

Practice proofs:
\begin{prop*}[\S1.2, \#17]
    % Find the statement in WA-0.tex
\end{prop*}

\begin{proof} Let \textit{???}. Assume \textit{???}.
    % your proof goes here
    Therefore, \textit{???}.
\end{proof}

\begin{prop*}[\S1.2, \#18] 
    % Find the statement in WA-0.tex
\end{prop*}

\begin{proof}
    % your proof goes here
\end{proof}